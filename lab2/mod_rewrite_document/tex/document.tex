\documentclass{article}
\usepackage{hyperref}
\usepackage[MeX]{polski}
\usepackage{parskip}
\usepackage[utf8]{inputenc}
\usepackage{graphicx}
\usepackage{listings}
\title{Przyjazne URLe}
\author{\textsc{Paweł Sołtysiak I1-41A} \\ \texttt{psoltysiak@wi.zut.edu.pl}}

\usepackage{color}

\definecolor{mygreen}{rgb}{0,0.6,0}
\definecolor{mygray}{rgb}{0.5,0.5,0.5}
\definecolor{mymauve}{rgb}{0.58,0,0.82}
\lstset{ %
  backgroundcolor=\color{white},   % choose the background color; you must add \usepackage{color} or \usepackage{xcolor}
  basicstyle=\ttfamily,        % the size of the fonts that are used for the code
  breakatwhitespace=false,         % sets if automatic breaks should only happen at whitespace
  breaklines=true,                 % sets automatic line breaking
  captionpos=b,                    % sets the caption-position to bottom
  commentstyle=\color{mygreen},    % comment style
  escapeinside={\%*}{*)},          % if you want to add LaTeX within your code
  extendedchars=true,              % lets you use non-ASCII characters; for 8-bits encodings only, does not work with UTF-8
  frame=single,                    % adds a frame around the code
  keepspaces=true,                 % keeps spaces in text, useful for keeping indentation of code (possibly needs columns=flexible)
  keywordstyle=\color{blue},       % keyword style
  language=Octave,                 % the language of the code
  numbers=none,                    % where to put the line-numbers; possible values are (none, left, right)
  numbersep=5pt,                   % how far the line-numbers are from the code
  numberstyle=\tiny\color{mygray}, % the style that is used for the line-numbers
  rulecolor=\color{black},         % if not set, the frame-color may be changed on line-breaks within not-black text (e.g. comments (green here))
  showspaces=false,                % show spaces everywhere adding particular underscores; it overrides 'showstringspaces'
  showstringspaces=false,          % underline spaces within strings only
  showtabs=false,                  % show tabs within strings adding particular underscores
  stepnumber=2,                    % the step between two line-numbers. If it's 1, each line will be numbered
  stringstyle=\color{mymauve},     % string literal style
  tabsize=2,                       % sets default tabsize to 2 spaces
  title=\lstname                   % show the filename of files included with \lstinputlisting; also try caption instead of title
}

\begin{document}
\maketitle

\section{Mechanizm działania przyjaznych URL}
\subsection{Instalacja}
Do działania przyjaznych URL potrzebne jest zainstalowanie modułu \texttt{mod\_rewrite} do serwera Apache. W systemie ubuntu moduł mod\_rewrite jest instalowany razem z serwerem Apache, ale jest on domyślnie wyłączony.

\texttt{apt-get install apache2}

\texttt{sudo a2enmod rewrite}

\texttt{sudo service apache2 restart}

\subsection{Prosty przykład}
Moduł mod\_rewrite wykonuje translacje adresów URL. Przeglądarka wysyła żądanie dotyczące pliku o nazwie \texttt{adres.html}. Moduł \texttt{mod\_rewrite} powoduje, że serwer Apache, w odpowiedzi na żądanie o plik \texttt{adres.html} wyśle plik \texttt{index.php}. Na serwerze nie ma pliku o nazwie \texttt{adres.html}.

Następnie w tym samym folderze tworzymy plik o nazwie \texttt{.htaccess}. W pliku tym umieszczamy dwie dyrektywy konfiguracyjne \texttt{RewriteEngine} oraz \texttt{RewriteRule}: 

\begin{lstlisting}[caption=plik .htaccess]
RewriteEngine on
RewriteRule adres.html index.php
\end{lstlisting}

Pierwsza z nich włącza działanie modułu mod\_rewrite, a druga ustala wykonywaną translację. Podana reguła powoduje, że żądanie HTTP:

\texttt{GET /adres.html HTTP/1.1}

będzie obsługiwane identycznie, jak żądanie:

\texttt{GET /index.php HTTP/1.1}

Innymi słowy, w odpowiedzi na żądanie o plik \texttt{adres.html}, serwer Apache wyśle plik \texttt{index.php} (dokładniej: kod HTML wygenerowany przez skrypt index.php). 

\section{Mechanizm działania kontrolera}
Zadaniem kontrolera jest odbiór, przetworzenie oraz analiza danych wejściowych od użytkownika. W typowej aplikacji źródłami danych wejściowych będą klawiatura i mysz. Po przetworzeniu odebranych danych kontroler może wykonać następujące czynności:
\begin{itemize}
\item zmienić stan modelu,
\item odświeżyć widok,
\item przełączyć sterowanie na inny kontroler.

\end{itemize}
    
Każdy kontroler posiada bezpośrednie wskazania na określone modele i widoki, z którymi współpracuje, a jednocześnie w aplikacji może istnieć wiele kontrolerów. W danym momencie tylko jeden z nich steruje aplikacją.
\end{document}

